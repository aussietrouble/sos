\documentclass[12pt]{imsart}
\setattribute{infoline}{text}{file: \jobname.tex\ date: \today}
\def\abstractname{Summary}

\RequirePackage[OT1]{fontenc}
\RequirePackage{amsthm,amsmath,amsfonts,natbib}
%\RequirePackage[colorlinks,citecolor=blue,urlcolor=blue]{hyperref}
\RequirePackage{hypernat}
\usepackage{times}
%\usepackage[lite,subscriptcorrection,slantedGreek,nofontinfo]{mtpro2}
\bibliographystyle{abbrvnat}
\usepackage{algorithm}
\usepackage{algorithmic}
\usepackage{fullpage}
\usepackage{graphicx,url}

\bibpunct{(}{)}{;}{a}{,}{,}

% settings
%\pubyear{2006}
%\volume{0}
%\issue{0}
%\firstpage{1}
%\lastpage{8}

\startlocaldefs
\numberwithin{equation}{section}
\theoremstyle{plain}
\newtheorem{theorem}{Theorem}[section]
\newtheorem{corollary}{Corollary}[section]
\newtheorem{proposition}{Proposition}[section]
\newtheorem{lemma}{Lemma}[section]
\newtheoremstyle{remark}{\topsep}{\topsep}%
     {\normalfont}% Body font
     {}           % Indent amount (empty = no indent, \parindent = para indent)
     {\bfseries}  % Thm head font
     {.}          % Punctuation after thm head
     {.5em}       % Space after thm head (\newline = linebreak)
     {\thmname{#1}\thmnumber{ #2}\thmnote{ #3}}% Thm head spec
\theoremstyle{remark}
\newtheorem{remark}{Remark}[section]
\newtheorem{example}{Example}[section]
\newtheorem{assumption}{Assumption}[section]
\newtheorem{definition}{Definition}[section]
\endlocaldefs

\renewcommand{\baselinestretch}{1.1}
\setcounter{tocdepth}{2}
%\parskip12pt
\parindent15pt
\footskip30pt
%\large\normalsize
\def\comma{\unskip,~}
\def\truep{p^*}
\def\div{\|\,}
\long\def\comment#1{}
\def\reals{{\mathbb R}}
\def\P{{\mathbb P}}
\def\E{{\mathbb E}}
\def\supp{\mathop{\text{supp}\kern.2ex}}
\def\argmin{\mathop{\text{\rm arg\,min}}}
\def\arginf{\mathop{\text{\rm arg\,inf}}}
\def\argmax{\mathop{\text{\rm arg\,max}}}
\let\hat\widehat
\let\tilde\widetilde
\def\csd{${}^*$}
\def\mld{${}^\dag$}
\def\dos{${}^\ddag$}
\def\W{\widetilde Y}
\def\Z{\widetilde X}
\let\hat\widehat
\let\tilde\widetilde
\def\ds{\displaystyle}
\def\bs{\backslash}
\def\1{{(1)}}
\def\2{{(2)}}
\def\pn{{(n)}}
\def\ip{{(i)}}
\def\Xbar{\overline{X}}
\def\except{\backslash}
\def\npn{\mathop{\textit{NPN\,}}}
\def\npnsymbol{
  \pspicture(-.2,-.2)(.2,.2)
  \def\cw{.20}
  \cnode[fillstyle=none,linewidth=.5pt](0,0){8pt}{L1}
  \rput(0,0){\tiny\PHhide}
  \psline[linewidth=.5pt,linecolor=black]{-}(-\cw,-\cw)(\cw,\cw)
  \endpspicture
}
\def\i{{(i)}}
\def\cE{{\mathcal{C}}}
\def\cM{{\mathcal{M}}}
\def\cF{{\mathcal{F}}}
\def\cP{{\mathcal{P}}}
\def\cG{{\mathcal{G}}}
\def\M{{\mathcal{M}}}
\def\tr{\mathop{\text{tr}}}
\long\def\comment#1{}
\def\N{\textit{N}\kern.3ex}
\def\t{{\scriptstyle \top}}
\def\fs{\footnotesize}
\let\hat\widehat
\let\epsilon\varepsilon
\let\phi\varphi
\def\ccc{CCC}
\def\N{\mbox{\it N}\,}
\def\NPN{\mathop{\mbox{\it NPN}}\,}
\def\F{\mathcal{F}}
\def\T{\mathcal{T}}
\def\H{\mathcal{H}}
\def\S{S}
\def\X{\mathcal{F}}
\def\Cov{\mathop{\mathbb{C}\textrm{ov}}\kern.1ex}
\def\Var{\mathop{\mathbb{V}\textrm{ar}}\kern.1ex}
\def\had{\!\circ\!}
\def\PNL{X^v}
\def\Cost{C^v}
\def\Value{V^v}
\def\Fill{\iS}
\def\Impacted{\tilde S}
\def\uS{\tilde S}
\def\iS{S^v}
\def\Volume{\tilde N}
\def\uVolume{N}
\def\diag{\text{diag}}
\def\reals{{\mathbb R}}
\def\P{{\mathbb P}}
\def\E{{\mathbb E}}
\def\supp{\mathop{\text{supp}\kern.2ex}}
\def\argmin{\mathop{\text{arg\,min}\kern.2ex}}
\def\argmax{\mathop{\text{arg\,max}\kern.2ex}}
\let\hat\widehat
\let\tilde\widetilde
\def\csd{${}^*$}
\def\mld{${}^\dag$}
\def\dos{${}^\ddag$}
\def\W{\widetilde Y}
\def\Z{\widetilde X}
\def\given{\,|\,}
\def\C{\mathcal{C}}
\def\D{\mathcal{D}}
\def\M{\mathcal{M}}
%\def\N{\mathcal{N}}
\def\N{S}
\def\X{\mathcal{X}}
\def\tr{\mathop{\text{tr}}}
\def\ntr{\mathop{\text{tr}_n}}
\def\ptr{\mathop{\text{tr}_p}}
\def\s{\backslash}
\def\p{\partial}
\def\MS#1{\tilde{#1}}
%\def\MS#1{#1[\uS]}


\parskip12pt
\parindent0pt

\begin{document}

\begin{frontmatter}
%\centerline{\large\bf Log-SOS-Concave Density Estimation and Graphical
%  Modeling}
\end{frontmatter}

\par
%\section{Introduction}
%\label{intro}

%\section{Problem}
\stepcounter{section}

Here is an idea for how we can form a class of log-concave densities
for which the optimization will involve finite-dimensional
semidefinite programming.  This may lead to a
theoretically and practically tractable family 
of log-concave density estimators and graphical models.


To begin, skim Chapter 3, ``Convexity and SOS-Convexity'' in the MIT dissertation of 
Amir Ali Ahmadi, ``Algebraic Relaxations and Hardness Results
in Polynomial Optimization and Lyapunov Analysis,''
{\small \url{http://hdl.handle.net/1721.1/68434}}.
(Recall that Philippe Rigollet pointed us to this work.)

Now, define the family of log-sos-concave densities as 
\begin{equation}
p(x) \propto \exp\left(-q(x)\right)
\end{equation}
where $q(x)$ is an sos-convex polynomial.  
We can thus write $q(x) = \theta^T \phi(x)$ where $\phi(x)$ is a
basis of polynomials, e.g., $\phi(x) = (1,x_1,x_2,\ldots, x_p, x_1x_2,\ldots, x_p^{d})$. 

The statement that 
$\theta^T \phi_\theta(x)$ is an sos-convex function
means that $r_\theta(x,y) \equiv y^T H_\theta(x) y$ is an sos polynomial, 
where $H_\theta(x) = \nabla^2 \phi_\theta(x)$ is the Hessian,
and $y=(y_1,\ldots, y_p)$.  Equivalently, 
\begin{align}
y^T H_\theta(x) y & = z^T Q z \\
Q & \succeq 0
\end{align}
where $z$ represents the homogeneous polynomials in $(x,y)$ up to
order $d/2$. This property is equivalent to a semidefinite feasibility program.

Since a log-sos-concave density is log-concave, sampling is efficient.
Thus, the score function
\begin{equation}
\nabla \ell(\theta) = \frac{1}{n} \sum_{i=1}^n \phi(X_i) - \E_\theta\bigl(\phi(X)\bigr)
\end{equation}
can be efficiently approximated. Together with the computational 
tractability of semidefinite programming, this should
in principle lead to a tractable family of densities and
graphical models. 

The book ``Moments, Positive Polynomials and Their Applications,'' by Jean
Bernard Lasserre,
{\small\url{http://www.stat.uchicago.edu/~lekheng/courses/310/books/lasserre.pdf}}
may be a good resource for ideas and results relevant to this direction.

\end{document}
