%%%%%%%%%%%%% Packages
\usepackage{xspace,xcolor,enumerate}
\usepackage{amsmath,amsfonts,amssymb}
\usepackage{color,graphicx}
\usepackage{dsfont}


%%%%%%%%%%%%%%% Lengths
\setlength{\topmargin}{-0.5 in} \setlength{\oddsidemargin}{0 in}
\setlength{\evensidemargin}{0 in} \setlength{\textwidth}{6.5 in}
\setlength{\textheight}{8.5 in} \setlength{\headsep}{0.75 in}
\setlength{\parindent}{0 in} \setlength{\parskip}{0.05 in}


%%%%%%%%%%%%%%%%%% Theorem Environments
\newtheorem{theorem}{Theorem}[section]
\newtheorem{conjecture}[theorem]{Conjecture}
\newtheorem{definition}[theorem]{Definition}
\newtheorem{lemma}[theorem]{Lemma}
\newtheorem{remark}[theorem]{Remark}
\newtheorem{proposition}[theorem]{Proposition}
\newtheorem{corollary}[theorem]{Corollary}
\newtheorem{claim}[theorem]{Claim}
\newtheorem{fact}[theorem]{Fact}
\newtheorem{openprob}[theorem]{Open Problem}
\newtheorem{remk}[theorem]{Remark}
\newtheorem{example}[theorem]{Example}
\newtheorem{exercise}[theorem]{Exercise}
\newtheorem{apdxlemma}{Lemma}
\newtheorem{algorithm}[theorem]{Algorithm}
\newcommand{\question}[1]{{\sf [#1]\marginpar{?}} }



%%%%%%%%%%%%% Additional Macros for Citations %%%%%%%%%%%%%%%%
% Convention for citations is authors' initials followed by the year.
% For example, to cite a paper by Leighton and Maggs you would type
% \cite{LM89}, and to cite a paper by Strassen you would type \cite{S69}.
% (To avoid bibliography problems, for now we redefine the \cite command.)
% Also commands that create a suitable format for the reference list.
\renewcommand{\cite}[1]{[#1]}
\def\beginrefs{\begin{list}%
        {[\arabic{equation}]}{\usecounter{equation}
         \setlength{\leftmargin}{2.0truecm}\setlength{\labelsep}{0.4truecm}%
         \setlength{\labelwidth}{1.6truecm}}}
\def\endrefs{\end{list}}
\def\bibentry#1{\item[\hbox{[#1]}]}


%%%%%%%%%%%%%%%%% Proof Environments
\def\FullBox{\hbox{\vrule width 6pt height 6pt depth 0pt}}

\def\qed{\ifmmode\qquad\FullBox\else{\unskip\nobreak\hfil
\penalty50\hskip1em\null\nobreak\hfil\FullBox
\parfillskip=0pt\finalhyphendemerits=0\endgraf}\fi}

\def\qedsketch{\ifmmode\Box\else{\unskip\nobreak\hfil
\penalty50\hskip1em\null\nobreak\hfil$\Box$
\parfillskip=0pt\finalhyphendemerits=0\endgraf}\fi}

\newenvironment{proof}{\begin{trivlist} \item {\bf Proof:~~}}
   {\qed\end{trivlist}}

\newenvironment{proofsketch}{\begin{trivlist} \item {\bf
Proof Sketch:~~}}
  {\qedsketch\end{trivlist}}

\newenvironment{proofof}[1]{\begin{trivlist} \item {\bf Proof
#1:~~}}
  {\qed\end{trivlist}}

\newenvironment{claimproof}{\begin{quotation} \noindent
{\bf Proof of claim:~~}}{\qedsketch\end{quotation}}

%%%%%%%%%% Symbols and Fonts
\def\from{:}
\def\to{\rightarrow}
\def\eps{\varepsilon}
\def\epsilon{\varepsilon}
\def\e{\epsilon}
\def\eps{\epsilon}
\def\d{\delta}
\def\phi{\varphi}
\def\cal{\mathcal}
\def\xor{\oplus}
\def\ra{\rightarrow}
\def\implies{\Rightarrow}
\def\psdgeq{\succeq} 
\renewcommand{\bar}{\overline} 
\newcommand{\ol}{\overline}

\def\bull{\vrule height .9ex width .8ex depth -.1ex }

%%%%%%%%%%%%%%%%%%%%%% Text Macros
\newcommand{\ie}{i.e.,\xspace}
\newcommand{\eg}{e.g.,\xspace}
\newcommand{\etal}{et al.\xspace}
\newcommand{\cf}{{\it cf.,}}

%%%%%%%%%%%%%%%%%%%%% Punctuation at the end of a displayed formula
\newcommand{\mper}{\,.}
\newcommand{\mcom}{\,,}

%%%%%%%%%%%%%%%%%%%%%% Number Sets
\newcommand{\R}{{\mathbb R}}
\newcommand{\E}{{\mathbb E}}
\newcommand{\C}{{\mathbb C}}
\newcommand{\N}{{\mathbb{N}}}
\newcommand{\Z}{{\mathbb Z}}
\newcommand{\F}{{\mathbb F}}
\newcommand{\zo}{\{0,1\}}
\newcommand{\GF}{\mathrm{GF}}
\newcommand{\FF}{{\mathbb F}}
\newcommand{\Real}{{\mathbb R}}

\newcommand{\B}{\{0,1\}\xspace}
\newcommand{\pmone}{\{-1,1\}\xspace}

\newcommand{\indicator}[1]{\mathds{1}_{\{#1\}}}

%%%%%%%%%%%%%%%%%%%%% Random Variables and Probability
\newcommand{\Esymb}{\mathbb{E}}
\newcommand{\Psymb}{\mathbb{P}}
\newcommand{\Vsymb}{\mathsf{Var}}

\DeclareMathOperator*{\ExpOp}{\Esymb}
\DeclareMathOperator*{\VarOp}{\Vsymb}
\DeclareMathOperator*{\ProbOp}{\Psymb}
\renewcommand{\Pr}{\ProbOp}

\newcommand{\prob}[1]{\Pr\left[{#1}\right]}
\newcommand{\Prob}[2]{\Pr_{{#1}}\left[{#2}\right]}
\newcommand{\ex}[1]{\ExpOp\left[{#1}\right]}
\newcommand{\Ex}[2]{\ExpOp_{{#1}}\left[{#2}\right]}
\newcommand{\var}[1]{\VarOp\left[{#1}\right]}
\newcommand{\Var}[2]{\VarOp_{{#1}}\left[{#2}\right]}


\newcommand{\conv}[1]{\mathrm{conv}\inparen{#1}}
\newcommand{\given}{\;\middle\vert\;}


%%%%%%%%%% Standard Normal Distribution
\newcommand{\gauss}[2]{{\cal N(#1, #2)}}

%%%%%%%%%%%% Fractions
% commands for fractions 
\usepackage{nicefrac}
% poor man's fraction
\newcommand{\flatfrac}[2]{#1/#2}
\newcommand{\varflatfrac}[2]{#1\textfractionsolidus#2}

\let\nfrac=\nicefrac
\let\ffrac=\flatfrac
% similar commands: tfrac,dfrac

\newcommand{\half}{\nicefrac12}
\newcommand{\onequarter}{\nicefrac14}
\newcommand{\threequarter}{\nicefrac34}

%%%%%%%%%%%%%%%%%% Vectors and Reals
\newcommand{\abs}[1]{\ensuremath{\left\lvert #1 \right\rvert}}
\newcommand{\smallabs}[1]{\ensuremath{\lvert #1 \rvert}}
%
\newcommand{\norm}[1]{\ensuremath{\left\lVert #1 \right\rVert}}
\newcommand{\smallnorm}[1]{\ensuremath{\lVert #1 \rVert}}
%
\newcommand{\mydot}[2]{\ensuremath{\left\langle #1, #2 \right\rangle}}
\newcommand{\mysmalldot}[2]{\ensuremath{\langle #1, #2 \rangle}}
\newcommand{\ip}[1]{\left\langle #1 \right\rangle}

%%%%%%%%%%% Vectors
\def\bfx{{\bf x}}
\def\bfy{{\bf y}}
\def\bfz{{\bf z}}
\def\bfu{{\bf u}}
\def\bfv{{\bf v}}
\def\bfw{{\bf w}}
\def\bfa{{\bf a}}
\def\bfb{{\bf b}}
\def\bfc{{\bf c}}
\def\bff{{\bf f}}
\def\bfg{{\bf g}}
\def\bfp{{\bf p}}
\def\bfq{{\bf q}}
\def\bfX{{\bf X}}
\def\bfY{{\bf Y}}
\def\bfZ{{\bf Z}}
\def\bfU{{\bf U}}
\def\bfV{{\bf V}}
\def\bfalpha{{\boldsymbol\alpha}}
\def\bfbeta{{\boldsymbol\beta}}
\def\bfgamma{{\boldsymbol\gamma}}
\def\bftheta{{\boldsymbol\theta}}

\newcommand{\zero}{\mathbf 0}
\newcommand{\one}{{\mathbf{1}}}
\newcommand{\zeroone}{{0/1}\xspace}
\newcommand{\minusoneone}{{-1/1}\xspace}

\newcommand{\yes}{{\sf Yes}\xspace}
\newcommand{\no}{{\sf No}\xspace}

%%%%%%%%%%%%%%%%%%%%% Linear Algebra
\DeclareMathOperator\Span{Span}
\DeclareMathOperator\kernel{ker}
\DeclareMathOperator\im{Im}
\DeclareMathOperator\image{Im}
\DeclareMathOperator\tr{tr}
\DeclareMathOperator\trace{tr}
\newcommand{\pd}{\succ} % positive definite
\newcommand{\psd}{\succeq} % positive semidefinite e.g. Q \psd 0


%%%%%%%%%%%%%%%%%%%%%% General Useful Macros
\newfont{\inhead}{eufm10 scaled\magstep1}
\newcommand{\deffont}{\sf}

\newcommand{\calA}{{\cal A}}
\newcommand{\calB}{{\cal B}}
\newcommand{\calC}{{\cal C}}
\newcommand{\calG}{{\cal G}}
\newcommand{\calL}{{\cal L}}
\newcommand{\calU}{{\cal U}}
\newcommand{\calP}{{\cal P}}
\newcommand{\poly}{{\mathrm{poly}}}
\newcommand{\polylog}{{\mathrm{polylog}}}
\newcommand{\loglog}{{\mathop{\mathrm{loglog}}}}
\newcommand{\suchthat}{{\;\; : \;\;}}
\newcommand{\getsr}{\mathbin{\stackrel{\mbox{\tiny R}}{\gets}}}


%%%%%%%%%% Operators
\DeclareMathOperator\supp{Supp}
\newcommand{\argmax}{\mathrm{argmax}}

%%%%%%%%%%%%%%%%%%% Complexity Classes
\newcommand{\classfont}[1]{\textsf{#1}}
\newcommand{\coclass}[1]{\mathbf{co\mbox{-}#1}} 
\newcommand{\BPP}{\classfont{BPP}\xspace}
\newcommand{\classP}{\classfont{P}\xspace}
\newcommand{\NP}{\classfont{NP}\xspace}
\newcommand{\coNP}{\classfont{coNP}\xspace}
\newcommand{\nphard}{\classfont{NP}-hard\xspace}

%%%%%%%%%%%%%%%%%%%% Problems
% use texorpdfstring to avoid problems with hyperref (can use problem
% macros also in headings
\newcommand{\problemmacro}[1]{\textsf{#1}}

\newcommand{\maxcsp}{\problemmacro{MAX-CSP}\xspace}
\newcommand{\maxkcsp}{\problemmacro{MAX k-CSP}\xspace}
\newcommand{\maxthreesat}{\problemmacro{MAX 3-SAT}\xspace}
\newcommand{\maxthreexor}{\problemmacro{MAX 3-XOR}\xspace}
\newcommand{\maxkxor}{\problemmacro{MAX k-XOR}\xspace}
\newcommand{\maxcut}{\problemmacro{MAX-CUT}\xspace}

%%%%%%%%%%%%%%%%%%%%%%%%% Enclosures
\newcommand{\inparen}[1]{\left(#1\right)}             %\inparen{x+y}  is (x+y)
\newcommand{\inbraces}[1]{\left\{#1\right\}}           %\inbrace{x+y}  is {x+y}
\newcommand{\insquare}[1]{\left[#1\right]}             %\insquare{x+y}  is [x+y]
\newcommand{\inangle}[1]{\left\langle#1\right\rangle} %\inangle{A}    is <A>


%%%%%%%%%%%%%%%%%%%% Names
\newcommand{\Erdos}{Erd\H{o}s\xspace}
\newcommand{\Renyi}{R\'enyi\xspace}
\newcommand{\Lovasz}{Lov\'asz\xspace}
\newcommand{\Juhasz}{Juh\'asz\xspace}
\newcommand{\Bollobas}{Bollob\'as\xspace}
\newcommand{\Furedi}{F\"uredi\xspace}
\newcommand{\Komlos}{Koml\'os\xspace}
\newcommand{\Luczak}{\L uczak\xspace}
\newcommand{\Kucera}{Ku\v{c}era\xspace}
\newcommand{\Szemeredi}{Szemer\'edi\xspace}
\newcommand{\Chvatal}{Chv\'atal}
\newcommand{\Holder}{H\"{o}lder}
\newcommand{\Plunnecke}{Pl\"unnecke}


%%%%%%%%%%%%%%%%%%%%%%%%%%%%%%%%%%%%%%%%%%%%%%%%%%%%%%%%%%%%%%%%%%%%%%%%%%%
%%%%%%%%%%%%%%%%%%%%%%%%%%%%%%%%%%%%%%%%%%%%%%%%%%%%%%%%%%%%%%%%%%%%%%%%%%%

\newlength{\tpush}
\setlength{\tpush}{2\headheight}
\addtolength{\tpush}{\headsep}


\newcommand{\titlebox}[5]
{
   \noindent
   \begin{center}
   \framebox{
      \vbox{\vspace{2mm}
    \hbox to 6.28in { {\bf #1 \hfill #2 } }
       \vspace{4mm}
       \hbox to 6.28in { {\Large \hfill #3 \hfill} }
       \vspace{2mm}
       \hbox to 6.28in { {#4 \hfill #5} }
      \vspace{2mm}}
   }
   \end{center}
   \vspace*{4mm}
}


\newcommand{\handout}[3]
{
\noindent\vspace*{-\tpush}\newline
\parbox{\textwidth}
{
{
% TTI Chicago  \hfill Handout N#1 \newline
\coursenum : \coursename \hfill #2 \newline
\courseprof \newline
\mbox{}\hrulefill\mbox{}}\vspace*{1ex}\mbox{}\newline
\bigskip
\begin{center}{\Large\bf #1}\end{center}
\bigskip
}
}


