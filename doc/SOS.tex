%%%%%%%%%%%%%%%%%%%%%%%%%%%%%%%%%%%%%%%%%%%%%%%%%%
% Some commands and examples in this template have been taken 
% from the template for CS 271 by Alistair Sinclair at UC Berkeley
%%%%%%%%%%%%%%%%%%%%%%%%%%%%%%%%%%%%%%%%%%%%%%%%%%

\documentclass[11pt]{article}

%% File containing various commands and macros
%%%%%%%%%%%%% Packages
\usepackage{xspace,xcolor,enumerate}
\usepackage{amsmath,amsfonts,amssymb}
\usepackage{color,graphicx}
\usepackage{dsfont}


%%%%%%%%%%%%%%% Lengths
\setlength{\topmargin}{-0.5 in} \setlength{\oddsidemargin}{0 in}
\setlength{\evensidemargin}{0 in} \setlength{\textwidth}{6.5 in}
\setlength{\textheight}{8.5 in} \setlength{\headsep}{0.75 in}
\setlength{\parindent}{0 in} \setlength{\parskip}{0.05 in}


%%%%%%%%%%%%%%%%%% Theorem Environments
\newtheorem{theorem}{Theorem}[section]
\newtheorem{conjecture}[theorem]{Conjecture}
\newtheorem{definition}[theorem]{Definition}
\newtheorem{lemma}[theorem]{Lemma}
\newtheorem{remark}[theorem]{Remark}
\newtheorem{proposition}[theorem]{Proposition}
\newtheorem{corollary}[theorem]{Corollary}
\newtheorem{claim}[theorem]{Claim}
\newtheorem{fact}[theorem]{Fact}
\newtheorem{openprob}[theorem]{Open Problem}
\newtheorem{remk}[theorem]{Remark}
\newtheorem{example}[theorem]{Example}
\newtheorem{exercise}[theorem]{Exercise}
\newtheorem{apdxlemma}{Lemma}
\newtheorem{algorithm}[theorem]{Algorithm}
\newcommand{\question}[1]{{\sf [#1]\marginpar{?}} }



%%%%%%%%%%%%% Additional Macros for Citations %%%%%%%%%%%%%%%%
% Convention for citations is authors' initials followed by the year.
% For example, to cite a paper by Leighton and Maggs you would type
% \cite{LM89}, and to cite a paper by Strassen you would type \cite{S69}.
% (To avoid bibliography problems, for now we redefine the \cite command.)
% Also commands that create a suitable format for the reference list.
\renewcommand{\cite}[1]{[#1]}
\def\beginrefs{\begin{list}%
        {[\arabic{equation}]}{\usecounter{equation}
         \setlength{\leftmargin}{2.0truecm}\setlength{\labelsep}{0.4truecm}%
         \setlength{\labelwidth}{1.6truecm}}}
\def\endrefs{\end{list}}
\def\bibentry#1{\item[\hbox{[#1]}]}


%%%%%%%%%%%%%%%%% Proof Environments
\def\FullBox{\hbox{\vrule width 6pt height 6pt depth 0pt}}

\def\qed{\ifmmode\qquad\FullBox\else{\unskip\nobreak\hfil
\penalty50\hskip1em\null\nobreak\hfil\FullBox
\parfillskip=0pt\finalhyphendemerits=0\endgraf}\fi}

\def\qedsketch{\ifmmode\Box\else{\unskip\nobreak\hfil
\penalty50\hskip1em\null\nobreak\hfil$\Box$
\parfillskip=0pt\finalhyphendemerits=0\endgraf}\fi}

\newenvironment{proof}{\begin{trivlist} \item {\bf Proof:~~}}
   {\qed\end{trivlist}}

\newenvironment{proofsketch}{\begin{trivlist} \item {\bf
Proof Sketch:~~}}
  {\qedsketch\end{trivlist}}

\newenvironment{proofof}[1]{\begin{trivlist} \item {\bf Proof
#1:~~}}
  {\qed\end{trivlist}}

\newenvironment{claimproof}{\begin{quotation} \noindent
{\bf Proof of claim:~~}}{\qedsketch\end{quotation}}

%%%%%%%%%% Symbols and Fonts
\def\from{:}
\def\to{\rightarrow}
\def\eps{\varepsilon}
\def\epsilon{\varepsilon}
\def\e{\epsilon}
\def\eps{\epsilon}
\def\d{\delta}
\def\phi{\varphi}
\def\cal{\mathcal}
\def\xor{\oplus}
\def\ra{\rightarrow}
\def\implies{\Rightarrow}
\def\psdgeq{\succeq} 
\renewcommand{\bar}{\overline} 
\newcommand{\ol}{\overline}

\def\bull{\vrule height .9ex width .8ex depth -.1ex }

%%%%%%%%%%%%%%%%%%%%%% Text Macros
\newcommand{\ie}{i.e.,\xspace}
\newcommand{\eg}{e.g.,\xspace}
\newcommand{\etal}{et al.\xspace}
\newcommand{\cf}{{\it cf.,}}

%%%%%%%%%%%%%%%%%%%%% Punctuation at the end of a displayed formula
\newcommand{\mper}{\,.}
\newcommand{\mcom}{\,,}

%%%%%%%%%%%%%%%%%%%%%% Number Sets
\newcommand{\R}{{\mathbb R}}
\newcommand{\E}{{\mathbb E}}
\newcommand{\C}{{\mathbb C}}
\newcommand{\N}{{\mathbb{N}}}
\newcommand{\Z}{{\mathbb Z}}
\newcommand{\F}{{\mathbb F}}
\newcommand{\zo}{\{0,1\}}
\newcommand{\GF}{\mathrm{GF}}
\newcommand{\FF}{{\mathbb F}}
\newcommand{\Real}{{\mathbb R}}

\newcommand{\B}{\{0,1\}\xspace}
\newcommand{\pmone}{\{-1,1\}\xspace}

\newcommand{\indicator}[1]{\mathds{1}_{\{#1\}}}

%%%%%%%%%%%%%%%%%%%%% Random Variables and Probability
\newcommand{\Esymb}{\mathbb{E}}
\newcommand{\Psymb}{\mathbb{P}}
\newcommand{\Vsymb}{\mathsf{Var}}

\DeclareMathOperator*{\ExpOp}{\Esymb}
\DeclareMathOperator*{\VarOp}{\Vsymb}
\DeclareMathOperator*{\ProbOp}{\Psymb}
\renewcommand{\Pr}{\ProbOp}

\newcommand{\prob}[1]{\Pr\left[{#1}\right]}
\newcommand{\Prob}[2]{\Pr_{{#1}}\left[{#2}\right]}
\newcommand{\ex}[1]{\ExpOp\left[{#1}\right]}
\newcommand{\Ex}[2]{\ExpOp_{{#1}}\left[{#2}\right]}
\newcommand{\var}[1]{\VarOp\left[{#1}\right]}
\newcommand{\Var}[2]{\VarOp_{{#1}}\left[{#2}\right]}


\newcommand{\conv}[1]{\mathrm{conv}\inparen{#1}}
\newcommand{\given}{\;\middle\vert\;}


%%%%%%%%%% Standard Normal Distribution
\newcommand{\gauss}[2]{{\cal N(#1, #2)}}

%%%%%%%%%%%% Fractions
% commands for fractions 
\usepackage{nicefrac}
% poor man's fraction
\newcommand{\flatfrac}[2]{#1/#2}
\newcommand{\varflatfrac}[2]{#1\textfractionsolidus#2}

\let\nfrac=\nicefrac
\let\ffrac=\flatfrac
% similar commands: tfrac,dfrac

\newcommand{\half}{\nicefrac12}
\newcommand{\onequarter}{\nicefrac14}
\newcommand{\threequarter}{\nicefrac34}

%%%%%%%%%%%%%%%%%% Vectors and Reals
\newcommand{\abs}[1]{\ensuremath{\left\lvert #1 \right\rvert}}
\newcommand{\smallabs}[1]{\ensuremath{\lvert #1 \rvert}}
%
\newcommand{\norm}[1]{\ensuremath{\left\lVert #1 \right\rVert}}
\newcommand{\smallnorm}[1]{\ensuremath{\lVert #1 \rVert}}
%
\newcommand{\mydot}[2]{\ensuremath{\left\langle #1, #2 \right\rangle}}
\newcommand{\mysmalldot}[2]{\ensuremath{\langle #1, #2 \rangle}}
\newcommand{\ip}[1]{\left\langle #1 \right\rangle}

%%%%%%%%%%% Vectors
\def\bfx{{\bf x}}
\def\bfy{{\bf y}}
\def\bfz{{\bf z}}
\def\bfu{{\bf u}}
\def\bfv{{\bf v}}
\def\bfw{{\bf w}}
\def\bfa{{\bf a}}
\def\bfb{{\bf b}}
\def\bfc{{\bf c}}
\def\bff{{\bf f}}
\def\bfg{{\bf g}}
\def\bfp{{\bf p}}
\def\bfq{{\bf q}}
\def\bfX{{\bf X}}
\def\bfY{{\bf Y}}
\def\bfZ{{\bf Z}}
\def\bfU{{\bf U}}
\def\bfV{{\bf V}}
\def\bfalpha{{\boldsymbol\alpha}}
\def\bfbeta{{\boldsymbol\beta}}
\def\bfgamma{{\boldsymbol\gamma}}
\def\bftheta{{\boldsymbol\theta}}

\newcommand{\zero}{\mathbf 0}
\newcommand{\one}{{\mathbf{1}}}
\newcommand{\zeroone}{{0/1}\xspace}
\newcommand{\minusoneone}{{-1/1}\xspace}

\newcommand{\yes}{{\sf Yes}\xspace}
\newcommand{\no}{{\sf No}\xspace}

%%%%%%%%%%%%%%%%%%%%% Linear Algebra
\DeclareMathOperator\Span{Span}
\DeclareMathOperator\kernel{ker}
\DeclareMathOperator\im{Im}
\DeclareMathOperator\image{Im}
\DeclareMathOperator\tr{tr}
\DeclareMathOperator\trace{tr}
\newcommand{\pd}{\succ} % positive definite
\newcommand{\psd}{\succeq} % positive semidefinite e.g. Q \psd 0


%%%%%%%%%%%%%%%%%%%%%% General Useful Macros
\newfont{\inhead}{eufm10 scaled\magstep1}
\newcommand{\deffont}{\sf}

\newcommand{\calA}{{\cal A}}
\newcommand{\calB}{{\cal B}}
\newcommand{\calC}{{\cal C}}
\newcommand{\calG}{{\cal G}}
\newcommand{\calL}{{\cal L}}
\newcommand{\calU}{{\cal U}}
\newcommand{\calP}{{\cal P}}
\newcommand{\poly}{{\mathrm{poly}}}
\newcommand{\polylog}{{\mathrm{polylog}}}
\newcommand{\loglog}{{\mathop{\mathrm{loglog}}}}
\newcommand{\suchthat}{{\;\; : \;\;}}
\newcommand{\getsr}{\mathbin{\stackrel{\mbox{\tiny R}}{\gets}}}


%%%%%%%%%% Operators
\DeclareMathOperator\supp{Supp}
\newcommand{\argmax}{\mathrm{argmax}}

%%%%%%%%%%%%%%%%%%% Complexity Classes
\newcommand{\classfont}[1]{\textsf{#1}}
\newcommand{\coclass}[1]{\mathbf{co\mbox{-}#1}} 
\newcommand{\BPP}{\classfont{BPP}\xspace}
\newcommand{\classP}{\classfont{P}\xspace}
\newcommand{\NP}{\classfont{NP}\xspace}
\newcommand{\coNP}{\classfont{coNP}\xspace}
\newcommand{\nphard}{\classfont{NP}-hard\xspace}

%%%%%%%%%%%%%%%%%%%% Problems
% use texorpdfstring to avoid problems with hyperref (can use problem
% macros also in headings
\newcommand{\problemmacro}[1]{\textsf{#1}}

\newcommand{\maxcsp}{\problemmacro{MAX-CSP}\xspace}
\newcommand{\maxkcsp}{\problemmacro{MAX k-CSP}\xspace}
\newcommand{\maxthreesat}{\problemmacro{MAX 3-SAT}\xspace}
\newcommand{\maxthreexor}{\problemmacro{MAX 3-XOR}\xspace}
\newcommand{\maxkxor}{\problemmacro{MAX k-XOR}\xspace}
\newcommand{\maxcut}{\problemmacro{MAX-CUT}\xspace}

%%%%%%%%%%%%%%%%%%%%%%%%% Enclosures
\newcommand{\inparen}[1]{\left(#1\right)}             %\inparen{x+y}  is (x+y)
\newcommand{\inbraces}[1]{\left\{#1\right\}}           %\inbrace{x+y}  is {x+y}
\newcommand{\insquare}[1]{\left[#1\right]}             %\insquare{x+y}  is [x+y]
\newcommand{\inangle}[1]{\left\langle#1\right\rangle} %\inangle{A}    is <A>


%%%%%%%%%%%%%%%%%%%% Names
\newcommand{\Erdos}{Erd\H{o}s\xspace}
\newcommand{\Renyi}{R\'enyi\xspace}
\newcommand{\Lovasz}{Lov\'asz\xspace}
\newcommand{\Juhasz}{Juh\'asz\xspace}
\newcommand{\Bollobas}{Bollob\'as\xspace}
\newcommand{\Furedi}{F\"uredi\xspace}
\newcommand{\Komlos}{Koml\'os\xspace}
\newcommand{\Luczak}{\L uczak\xspace}
\newcommand{\Kucera}{Ku\v{c}era\xspace}
\newcommand{\Szemeredi}{Szemer\'edi\xspace}
\newcommand{\Chvatal}{Chv\'atal}
\newcommand{\Holder}{H\"{o}lder}
\newcommand{\Plunnecke}{Pl\"unnecke}


%%%%%%%%%%%%%%%%%%%%%%%%%%%%%%%%%%%%%%%%%%%%%%%%%%%%%%%%%%%%%%%%%%%%%%%%%%%
%%%%%%%%%%%%%%%%%%%%%%%%%%%%%%%%%%%%%%%%%%%%%%%%%%%%%%%%%%%%%%%%%%%%%%%%%%%

\newlength{\tpush}
\setlength{\tpush}{2\headheight}
\addtolength{\tpush}{\headsep}


\newcommand{\titlebox}[5]
{
   \noindent
   \begin{center}
   \framebox{
      \vbox{\vspace{2mm}
    \hbox to 6.28in { {\bf #1 \hfill #2 } }
       \vspace{4mm}
       \hbox to 6.28in { {\Large \hfill #3 \hfill} }
       \vspace{2mm}
       \hbox to 6.28in { {#4 \hfill #5} }
      \vspace{2mm}}
   }
   \end{center}
   \vspace*{4mm}
}


\newcommand{\handout}[3]
{
\noindent\vspace*{-\tpush}\newline
\parbox{\textwidth}
{
{
% TTI Chicago  \hfill Handout N#1 \newline
\coursenum : \coursename \hfill #2 \newline
\courseprof \newline
\mbox{}\hrulefill\mbox{}}\vspace*{1ex}\mbox{}\newline
\bigskip
\begin{center}{\Large\bf #1}\end{center}
\bigskip
}
}





%%% Define any additional macros here
\newcommand{\ds}{\displaystyle}
\usepackage{tikz}

%%% Optimization problem (Use as a template if there are multiple constraints)
%%% Usage: \optprob{<minimize or maximize>}{<variables>}{<objective>}{<constraint>}{<reference label>}
\newcommand{\optprob}[5]
{
	\begin{equation}\label{#5}
	\begin{aligned}
	& \underset{#2}{\text{#1}}
	& & #3 \\
	& \text{\quad\ s.t.}
	& & #4
	\end{aligned}
	\end{equation}
}

\begin{document}
%% \titlebox{<top-left>}{<top-right>}{<title>}{<bottom-left>}{<bottom-right>}
\titlebox{Research Notes}{}{Nonparametric Estimation using SOS-Convexity}{Prof. John Lafferty}{Students: YJ Choe, Max Cytrynbaum, Wei Hu}

\section{Introduction}

(YJ will write out this section summerizing our first-day discussion when he has time.)

\clearpage

\section{SOS-Convex Regression}

Given $\{(\bfx_i, y_i)\}_{i=1}^n$, where $\bfx_i \in \R^p$ and $y_i \in \R$ for $i = 1, \dotsc, n$, recall that we have the equivalence between the following optimization problems:

\optprob{minimize}{}{\sum_{i=1}^n\inparen{y_i - f(\bfx_i)}^2}{\text{$f$ is convex.}}{cvx-reg}

\optprob{minimize}{\bfz, \bfbeta}{\sum_{i=1}^n\inparen{y_i - z_i}^2}{z_j \geq z_i + \bfbeta_i^T(\bfx_j - \bfx_i) \qquad\forall\ i, j = 1, \dotsc, n.}{cvx-reg-qp}

In particular, we can reduce the infinite-dimensional problem \eqref{cvx-reg} into a finite-dimensional quadratic program (QP) \eqref{cvx-reg-qp}, which can be efficiently solved. The solution to \eqref{cvx-reg-qp} can be viewed as a piecewise-linear convex function. 

Here, we attempt to derive the analogous equivalence, i.e. find an equivalent convex optimization problem to the following infinite-dimensional problem:
\optprob{minimize}{}{\sum_{i=1}^n\inparen{y_i - f(\bfx_i)}^2}{\text{$f$ is an SOS-convex polynomial of degree $2d$.}}{sos-reg}

Denote the vector of basis monomials up to degree $d$ by $\bfv_d(\bfx) = (1, x_1, \dotsc, x_p, x_1x_2, \dotsc, x_p^d)$, where $\bfx = (x_1, \dotsc, x_p)$. Also, let $s = {2d + p \choose p}$ denote the length of this vector for degree $2d$. Then, we may replace $f$ with a coefficient vector $\bftheta \in \R^s$, such that
\[
f(\bfx) = \bftheta^T\bfv_{2d}(\bfx).
\]
Note the one-to-one correspondence between $f$ and $\bftheta$. Further, as done with the convex program, we introduce the auxiliary variable $\bfz = (z_1, \dotsc, z_n)$ so that 
\begin{equation}\label{aux}
f(\bfx_i) = \bftheta^T\bfv_{2d}(\bfx_i) = z_i \qquad\forall\ i = 1, \dotsc, n.
\end{equation}
We can write this more concisely by introducing the matrix
\[
V = V(\bfx_1, \dotsc, \bfx_n) = \begin{bmatrix}
\bfv_d(\bfx_1) \\
\vdots \\
\bfv_d(\bfx_n)
\end{bmatrix}_{n \times s}
\]
so that \eqref{aux} simply becomes
\begin{equation}\label{sos-poly}
V\bftheta = \bfz.
\end{equation}

So we have a linear constraint on the coefficient $\theta$ that is equivalent to saying that the polynomial interpolates the points $\{(\bfx_i, z_i)\}_{i=1}^n$. Analogously, we can rewrite the objective to be
\begin{equation}\label{sos-reg-obj2}
\sum_{i=1}^n(y_i - z_i)^2 = \norm{\bfy - \bfz}^2
\end{equation}
where $\bfy = (y_1, \dotsc, y_n)$ and $\bfz = (z_1, \dotsc, z_n)$.


Now we want to rewrite the constraint that $f(\bfx) = \bftheta^T\bfv_d(\bfx)$ is SOS-convex. Recall that $p$ is SOS-convex if and only if
\begin{equation}\label{sos-cvx}
\bfu^TH_\bftheta(\bfx)\bfu = \bfv_d(\bfx,\bfu)^TQ\bfv_d(\bfx,\bfu)
\end{equation}
and
\begin{equation}\label{sos-cvx-psd}
Q \psd 0
\end{equation}
where $Q$ is a symmetric $r \times r$ matrix with $r = {d + 2p \choose 2p}$, $H_\theta(\bfx)$ is the Hessian polynomial matrix of $p(\bfx) = \bftheta^T\bfv_d(\bfx)$, and $\bfu = (u_1, \dotsc, u_p)$. 

Note first that \eqref{sos-cvx} is an equality in the space of \emph{polynomials} on $2p$ variables: $\bfx$ and $\bfu$. Also, \textbf{\eqref{sos-cvx-psd} suggests that the convex optimization program that we attempt to build is a semidefinite program (SDP).}

It is important to note that, \textbf{in this case where $f$ is a polynomial, the Hessian $H_\theta(\bfx)$ is easy to compute -- in fact, the coefficients of each entry of the Hessian are a linear function of the coefficient $\bftheta$ of the original polynomial $f$.} For a multinomial index $\bfalpha = (\alpha_1, \dotsc, \alpha_p) \in \N^p$ on $\bfx$ such that $\sum_{j=1}^p \alpha_j \leq d$, recall that the Hessian of $\bfx^\bfalpha$ with respect to $x_i$ and $x_j$ is
\[
\frac{\partial}{\partial x_i}\frac{\partial}{\partial x_j} \bfx^\bfalpha = \alpha_i\alpha_j \bfx^{\bfalpha_{i,j}'}
\]
where $\bfalpha_{i,j}' = (\alpha_1, \dotsc, \alpha_i-1, \dotsc, \alpha_j-1, \dotsc, \alpha_p)$ for $i \neq j$ and $\bfalpha_{i,i}' = (\alpha_1, \dotsc, \alpha_i-2, \dotsc, \alpha_p)$. Note that if $\alpha_i = 0$ for any $i$ then the Hessian is zero anyway.

\eqref{sos-cvx} is not a valid semidefinite constraint yet, because it is an equality between two polynomials. This means we want to equate the \emph{coefficients} of the two polynomials on $(\bfx, \bfy)$. Note first that
\begin{equation}\label{hessian}
\bfu^TH_\bftheta(\bfx)\bfu = \sum_{i,j=1}^p \inparen{\frac{\partial}{\partial x_i}\frac{\partial}{\partial x_j}f(\bfx)} u_iu_j.
\end{equation}
Then, there is at most one term (zero iff the partial derivative is zero) for each $\bfalpha = (\alpha_1, \dotsc, \alpha_p) \in \N^p$ such that $\sum_{k=1}^p \alpha_k \leq d$, and for each $i, j = 1, \dotsc, p$. For each of these cases, we give a multinomial index $\bfgamma_{\bfalpha, i, j} \in \N^{2p}$ on $(\bfx, \bfu)$, and $\bfgamma_{\bfalpha, i, j}$ will have a few specific properties: the first $p$ coordinates are precisely $\bfalpha_{i,j}'$, which sum up to at most $d-2$, and the last $p$ coordinates are all zero except at the $(p+i)$th and the $(p+j)$th coordinate (only at the $(p+i)$th if $i = j$). The first corresponds to the fact that the Hessian can have at most degree $d-2$, and the second to the fact that each term has exactly one degree on $u_i$ and $u_j$. Thus, we have
\begin{equation}\label{lhs}
\bfu^TH_\bftheta(\bfx)\bfu = \sum_\bfalpha \sum_{i,j} H_\bftheta(\bfalpha, i, j) (\bfx, \bfu)^{\bfgamma_{\bfalpha, i, j}}
\end{equation}
where $H_\bftheta(\bfalpha, i, j)$ represents the scalar coefficient for the term $(\bfx, \bfu)^{\bfgamma_{\bfalpha, i, j}}$.

Further, we can express the right-hand side in terms of their coordinates in the following way. First define the coordinate matrix $B_\bfgamma$ for each multi-index $\bfgamma \in \N^{2p}$ up to degree $2d$ such that
\[
\bfv_d(\bfx, \bfu)\bfv_d(\bfx, \bfu)^T = \sum_\bfgamma B_\bfgamma (\bfx, \bfu)^\bfgamma.
\]
Note that the matrices $B_\bfgamma$ are simply ``constants'', i.e. they do not depend on the data or the program variables. With this, the right-hand side becomes
\begin{align}\label{rhs}
\bfv_d(\bfx,\bfu)^TQ\bfv_d(\bfx,\bfu) &= \tr(Q\bfv_d(\bfx, \bfu)\bfv_d(\bfx, \bfu)^T) \notag\\
&= \ip{Q, \bfv_d(\bfx, \bfu)\bfv_d(\bfx, \bfu)^T} \notag\\
&= \ip{Q, \sum_\bfgamma B_\bfgamma (\bfx, \bfu)^\bfgamma} \notag\\
&= \sum_\bfgamma \ip{Q, B_\bfgamma}(\bfx, \bfu)^\bfgamma
\end{align}
where $\ip{A,B} = \tr(A^TB) = \sum_{i,j} A_{ij}B_{ij}$ is the matrix inner product. Note that $Q$ is symmetric.

Then, we can equate the coefficients of \eqref{lhs} and \eqref{rhs} to obtain:
\begin{align}
\ip{Q, B_{\bfgamma_{\bfalpha, i, j}}} &= H_\bftheta(\bfalpha, i, j) \qquad \forall\ \bfalpha, i, j \label{sos-cvx-coeff1}\\
\ip{Q, B_{\bfgamma}} &= 0 \qquad\qquad \text{for all other $\bfgamma$} \label{sos-cvx-coeff2}
\end{align}

Putting \eqref{sos-poly}, \eqref{sos-reg-obj2}, \eqref{sos-cvx-psd}, \eqref{sos-cvx-coeff1}, and \eqref{sos-cvx-coeff2} together, \eqref{sos-reg} can be restated as the following problem:

\begin{equation}\label{sos-reg2}
\begin{aligned}
& \underset{\bfz, \bftheta, Q}{\text{minimize}}
& & \norm{\bfy-\bfz}^2 \\
& \text{\quad\ s.t.}
& & V\bftheta = \bfz \\
& & & \langle{Q, B_{\bfgamma_{\bfalpha, i, j}}\rangle} = H_\bftheta(\bfalpha, i, j) \qquad \forall\ \bfalpha, i, j \\
& & & \langle{Q, B_{\bfgamma}\rangle} = 0 \qquad \text{for all other $\bfgamma$} \\
& & & Q \psd 0 
\end{aligned}
\end{equation}

\eqref{sos-reg2} is almost an SDP, except that the objective is quadratic. But in general, we can introduce another auxiliary variable $t$ to restate the problem as
\begin{equation}\label{sos-reg3}
\begin{aligned}
& \underset{t, \bfz, \bftheta, Q}{\text{minimize}}
& & t \\
& \text{\quad\ s.t.}
& & \norm{\bfy-\bfz}^2 \leq t \\
& & & V\bftheta = \bfz \\
& & & \langle{Q, B_{\bfgamma_{\bfalpha, i, j}}\rangle} = H_\bftheta(\bfalpha, i, j) \qquad \forall\ \bfalpha, i, j \\
& & & \langle{Q, B_{\bfgamma}\rangle} = 0 \qquad \text{for all other $\bfgamma$} \\
& & & Q \psd 0 
\end{aligned}
\end{equation}

Then, we are left with a quadratic inequality constraint. Fortunately, the following allows us to convert this into a semidefinite constraint.
\begin{fact}
For any $\bfx, \bfq \in \R^p$ and $r \in \R$, $\bfx^T\bfx + \bfq^T\bfx + r \leq 0$ if and only if $\begin{bmatrix} I & -\bfx \\ -\bfx^T & -\bfq^T\bfx - r \end{bmatrix} \psd 0$.
\end{fact}
\begin{proof}
For any $\bfy \in \R^p$ and $z \in \R$, 
\begin{align*}
\begin{bmatrix} \bfy^T & z \end{bmatrix} \begin{bmatrix} I & -\bfx \\ -\bfx^T & -\bfq^T\bfx - r \end{bmatrix} \begin{bmatrix} \bfy \\ z \end{bmatrix} &= \bfy^T\bfy -2z\bfx^T\bfy - z^2(\bfq^T\bfx+r) \\
&= \norm{\bfy - z\bfx}^2 - z^2(\bfx^T\bfx + \bfq^T\bfx + r).
\end{align*}
If $\bfx^T\bfx + \bfq^T\bfx + r \leq 0$, then this is nonnegative for all $\bfy \in \R^p$ and $z \in \R$. Otherwise, one can find $\bfy \in \R^p$ and $z \in \R$ such that this is strictly negative.
\end{proof}

Thus,
\begin{align*}
\norm{\bfy-\bfz}^2 \leq t &\iff \bfz^T\bfz - 2\bfy^T\bfz + (\bfy^T\bfy - t) \leq 0 \\
&\iff \begin{bmatrix} I & -\bfz \\ -\bfz^T & 2\bfy^T\bfz - \bfy^T\bfy + t \end{bmatrix} \psd 0.
\end{align*}
Note that the last relation is a linear matrix inequality (LMI), i.e. it says that a linear combination of symmetric matrices is positive semidefinite. 

Thus, we can now write \eqref{sos-reg3} into a semidefinite program:
\begin{equation}\label{sos-reg4}
\begin{aligned}
& \underset{t, \bfz, \bftheta, Q}{\text{minimize}}
& & t \\
& \text{\quad\ s.t.}
& & \begin{bmatrix} I & -\bfz \\ -\bfz^T & 2\bfy^T\bfz - \bfy^T\bfy + t \end{bmatrix} \psd 0 \\
& & & V\bftheta = \bfz \\
& & & \langle{Q, B_{\bfgamma_{\bfalpha, i, j}}\rangle} = H_\bftheta(\bfalpha, i, j) \qquad \forall\ \bfalpha, i, j \\
& & & \langle{Q, B_{\bfgamma}\rangle} = 0 \qquad \text{for all other $\bfgamma$} \\
& & & Q \psd 0 
\end{aligned}
\end{equation}
where the two semidefinite constraints can be restated -- if necessary -- into one semidefinite constraint
\[
\begin{bmatrix} I & -\bfz &  \\ -\bfz^T & 2\bfy^T\bfz - \bfy^T\bfy + t & \\ & & Q \end{bmatrix} \psd 0.
\]

Finally, note that the entire program depends on the degree of the SOS-convex polynomial that we started off with: $2d$. 

\subsection*{Further Questions}

\begin{enumerate}
\item What is the program size? Is it tractable?
\item Can the zero constraints be simplified?
\item For any given $d$, is the program feasible? What is the behavior of the objective $t_d$?
\item How can SDP hierarchy (e.g. by Lasserre) help choosing/removing $d$?
\end{enumerate}

\clearpage
\section{Convexity Pattern Problem}

We now consider a more restricted family of distributions that are hopefully more tractable and also have interesting applications.

With the familiar regression setting as in \eqref{cvx-reg}, first consider the additional constraint that $f$ is not only convex but also a function of only a few variables from $\bfx = (x_1, \dotsc, x_p)$. For example, we may have
\[
f(x_1, \dotsc, x_p) = f(x_1, x_2) \qquad \forall\ \bfx \in \R^p
\]
as one of the possibilities. 

In \cite{DCM}, Qi, Xu, and Lafferty shows a way to approximate the solution to the above problem \emph{additively}. Specifically, this is 
\optprob{minimize}{f_1, \dotsc, f_p}{\sum_{i=1}^n{(y_i - \sum_{j=1}^p f_j(x_{ij}))^2}}{f_1, \dotsc, f_p \text{ convex}}{cvx-additive-reg}
where $\bfx_i = (x_{i1}, \dotsc, x_{ip}) \in \R^p$. In other words, we have the model
\[
Y = \sum_{j=1}^p f_j(X_j) + \e
\]
in the population, with random variables $X = (X_1, \dotsc, X_p) \in \R^p$ and $Y \in \R$.

We can view this as a problem of \emph{sparsity patterns}, i.e. whether each variable is ``relevant'' ($f_j \not\equiv 0$) or not ($f_j \equiv 0$), and it is clear that there are $2^p$ sparsity patterns with $p$ variables.

\textbf{Here, we consider an analogous problem of choosing whether each $f_j$ is convex or concave.} Naturally, there are $2^p$ \emph{convexity patterns}. We can write this problem as the following optimization problem:
\begin{equation}\label{cvx-pattern}
\begin{aligned}
& \underset{\bfZ, \bff, \bfg}{\text{minimize}}
& & \sum_{i=1}^n \inparen{y_i - \sum_{j=1}^p\insquare{Z_jf_j(x_{ij}) + (1-Z_j)g_j(x_{ij})}}^2 \\
& \text{\quad\ s.t.}
& & Z_1, \dotsc, Z_p \in \{0,1\} \\
& & & f_1, \dotsc, f_p \text{ convex} \\
& & & g_1, \dotsc, g_p \text{ concave} 
\end{aligned}
\end{equation}
Note that $Z_1, \dotsc, Z_p$ are 0/1-boolean variables and $f_1, \dotsc, f_p, g_1, \dotsc, g_p$ are univariate functions.

In order to make the problem more tractable, we first give extra constraints: namely, that $f_1, \dotsc, f_p, g_1, \dotsc, g_p$ are \emph{polynomials}. It is important to note that a univariate polynomial is convex if and only if it is SOS-convex. [Problem: Is the set of convex polynomials dense in the set of convex functions? Is this relevant?] We can rewrite the program as follows:
\begin{equation}\label{cvx-pattern2}
\begin{aligned}
& \underset{\bfZ, \bff, \bfg}{\text{minimize}}
& & \sum_{i=1}^n \inparen{y_i - \sum_{j=1}^p\insquare{Z_jf_j(x_{ij}) + (1-Z_j)g_j(x_{ij})}}^2 \\
& \text{\quad\ s.t.}
& & Z_1, \dotsc, Z_p \in \{0,1\} \\
& & & f_1, \dotsc, f_p \text{ are (SOS-)convex polynomials of degree at most $d$} \\
& & & g_1, \dotsc, g_p \text{ are (SOS-)concave polynomials of degree at most $d$} 
\end{aligned}
\end{equation}
Using the similar trick as above, we hope to convert the constraints on $f_1, \dotsc, f_p, g_1, \dotsc, g_p$ into linear or semidefinite ones. 

A more important feature of this program is the use of 0-1 variables. It is well-known that, in general, solving a 0-1 integer linear program is NP-hard, and one of the standard procedures in theoretical computer science in dealing with this problem is to relax it such that the boolean constraint is replaced by $Z_1, \dotsc, Z_p \in [0,1]$, or equivalently the quadratic constraint $Z_j^2 - Z_j \leq 0\ \forall j=1, \dotsc, p$.

With this relaxation comes a family of LP/SDP hierarchies, such as the ones developed by Lov\'asz-Schrijver, Sherali-Adams, and Lasserre. [Prof. Madhur Tulsiani's Survey] These hierarchies are all a sequence of convex programs (LPs or SDPs) whose objective approaches the actual 0-1 solution. 

A good way to think about the hierarchies for 0-1 programs is to consider the $Z_j$'s the marginals of a distribution over a set of 0-1 solutions. Specifically, in the initial ``round'', consider $Z_j$ to be the marginal of the solution whose $j$th entry is 1 and all others are zero. Then, in consecutive rounds, the goal is to add the \emph{joint probabilities} between these variables -- in the $r$th round, we consider the joint random variables $Z_S$ for each $S \subseteq \{1, \dotsc, p\}$ such that $\abs{S} \leq r$. One can think of these ``big variables'' as $Z_S = \ex{\prod_{j \in S} Z_j}$, i.e. the probability that all variables in $S$ are 1.

Our hope is to use one of the hierarchies to solve a set of relaxations of \eqref{cvx-pattern2} that approximates the actual solution efficiently.

\clearpage
\section{Log-SOS-Concave Density Estimation}

%%%%%%%%%%%%%%%%%%%%%%%%%%%%%%%%%%%%%%%%%%%%
% I have actually re-defined the \cite{} command so that the
% citation will simply display as whatever you name it to be.
%
% There is no bibtex here that you need to run. If you say
% \cite{foo}, the lecture notes will simply display [foo].
%
% You don't need a .bib file. You will need to manually put in
% entries in the format below for whatever references you cite.
%%%%%%%%%%%%%%%%%%%%%%%%%%%%%%%%%%%%%%%%%%%%

\clearpage
\section*{References}
\beginrefs
\bibentry{BV}{\sc Boyd, S.} and {\sc Vandenberghe, L.} (2009).
{\it Convex Optimization}.
Cambridge University Press.
\bibentry{Lasserre}{\sc Lasserre, J. B.} (2009).
{\it Moments, Positive Polynomials and Their Applications}. Vol. 1. 
World Scientific.
\bibentry{DCM}{\sc Qi, Y.}, {\sc Xu, M.}, and {\sc Lafferty, J.} (2014).
{\it Learning High-Dimensional Concave Utility Functions for Discrete Choice Models}.
NIPS Submission.
\endrefs



% **** THIS ENDS THE EXAMPLES. DON'T DELETE THE FOLLOWING LINE:

\end{document}
