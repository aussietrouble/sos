\documentclass[11pt,reqno]{amsart}
\usepackage{amssymb,mathrsfs,color}
\usepackage{pinlabel}
\usepackage{graphicx}
\usepackage{graphics} 

\graphicspath{ {c:/users/mcytrynbaum/documents/rfigures/} {c:/users/mcytrynbaum/Desktop/package/} }
\DeclareGraphicsExtensions{.pdf,.png,.jpg}

%\usepackage[notcite,notref]{showkeys}

%\textwidth=15cm \textheight=21.5cm
%\oddsidemargin=0.5cm \evensidemargin=0.5cm
\usepackage{amsmath} % for all math functions and operations
\usepackage{amsfonts} % use this to write different scripts (e.g. Real nums, etc)
\usepackage{mathtools} %for other math stuff not included in packages above
\usepackage{amsthm} % in case you want the THM: COR: LEMMA: setup
\usepackage[top=1in,bottom=1in,left=1in,right=1in]{geometry} %for setting the margins

\setlength\parindent{0pt}

\newtheorem{thm}{Theorem}[section]
\newtheorem{lemma}[thm]{Lemma}
\newtheorem{prop}[thm]{Proposition}
\newtheorem{cor}[thm]{Corollary}
\theoremstyle{definition}
\newtheorem{defn}[thm]{Definition}
\newtheorem{examp}[thm]{Example}
\newtheorem{remark}[thm]{Remark}
\setcounter{equation}{0}
\numberwithin{equation}{section}

\newcommand{\prf}{\begin{proof}}
\newcommand{\eprf}{\end{proof}}
\newcommand{\lft}{\left(}
\newcommand{\rt}{\right)}
\newcommand{\be}{\beta}
\newcommand{\eps}{\epsilon}
\newcommand{\wh}{\widehat}
\newcommand{\wt}{\widetilde}
\newcommand{\al}{\alpha}
\newcommand{\bp}{\begin{pmatrix}}
\newcommand{\ep}{\end{pmatrix}}
\newcommand{\inv}{^{-1}}
\newcommand{\var}{\text{Var}}
\newcommand{\cov}{\text{Cov}}
\newcommand{\corr}{\text{Corr}}
\newcommand{\ssumi}{\sum_{i=1}^n}
\newcommand{\ssumj}{\sum_{j=1}^n}
\newcommand{\ssumk}{\sum_{k=1}^n}
\newcommand{\im}{\text{Im}}
\newcommand{\mc}{\mathcal}
\newcommand{\mr}{\mathbb{R}}
\newcommand{\ol}{\overline}
\newcommand{\ul}{\underline}
\newcommand{\prob}{\mathbb{P}}
\newcommand{\ital}{\emph}
\newcommand{\tb}{\textbf}
\newcommand{\pa}{\partial}
\newcommand{\dt}{\delta}
\newcommand{\tc}{t_{comm}}
\newcommand{\tp}{t_{patent}}
\newcommand{\et}{\eta}
\newcommand{\ov}{\overline}

\title{Statistical Properties of Log SOS-Concave Density Estimator}
\author{Max Cytrynbaum, Wei Hu}

\begin{document}
\maketitle

\section{Introduction}

In this note, we will describe an assortment of results that we think could eventually lead to a nice analysis of the statistical properties of our estimator. The structure of this note is as follows: \\

(i) We show that sos-convex polynomials are dense in continuous convex functions in the sense of $\| \cdot \|_{\infty, K}$, where $K$ is a fixed compact set. \\

(ii) Let $f_n$ denote the Cule estimator for $n$ data points (i.e. the max-likelihood log-concave estimator). We show that there exists a sequence of log sos-concave densities $p_n^m$ on $\mr^p$ (of potentially increasing associated degree) that converge to $f_n$ in distribution on $\mr^p$. \\

(iii) We show that this implies convergence of our objective function value to the likelihood value obtained by Cule's estimator $f_n$. We think that these facts will help us show that the estimator \emph{returned by our algorithm} converges to Cule's estimator in distribution. If this is true, then the density returned by our algorithm converges in distribution to the min KL-divergence estimator as we let $n, d \to \infty$ i.e. it is asymptotically optimal. \\

\section{Density of SOS-Convex Functions in Continuous Convex Functions}

Fix a compact set $K \subset \mr^p$ star-shaped with respect to $x_0 \in K$. All density results in this section are with respect to the norm $\| \cdot \|_{\infty, K}$. We begin by showing that sos-convex polynomials (SOSX) are dense in smooth, strictly convex functions on $K$ (SMSX). \\

Fix $f \in SMSX$. Let $H_f(x)$ denote this function's Hessian. For $x \in K$ we have $H_f(x) \succ 0$. Therefore, we can take a Cholesky factorization 

\[
H_f(x) = L_f(x)L_f(x)^T
\]

Note that since the component functions of $H_f$ are smooth, recursively all of the component functions of $L_f$ are in particular continuous (computing $L_{ij}(x)$ just involves adding, multiplying, and potentially taking the square root of other continuous functions). \\

By Stone-Weierstrass, $\mr[x]_p$ is dense in $C(K)$, therefore we can find polynomials $L_{ij}^m(x) \in \mr[x]_p$ with the property $\|L_{ij}^m(x) - L_{ij}(x)\|_{\infty} \to 0 $ as $m \to \infty$ (in particular we can get simultaneous uniform convergence of the $L_{ij}^m \to L_{ij}$). Define a matrix 

\[
H_f^m(x) = L_f^m(x)L_f^m(x)^T
\]

Then by the above work we have that 

\[
\|H_f^m - H_f\|_{\infty,K} \to 0 \qquad (m \to \infty )
\]

Note that for each $m$, $H_f^m(x)$ is a polynomial matrix over $\mr^p$ that factors as $H_f^m(x) = L_f^m(x)L_f^m(x)^T$, where the Cholesky factors \emph{are also polynomial matrices}. Suppose that we can construct a polynomial $p_m$ that has $H_f^m$ as its Hessian i.e. $H_{p_m} = H_f^m$. Then by construction $p_m$ is sos-convex. \\

By the above argument, we can find a sequence of polynomial matrices $H_f^m(x)$ such that 

\[
\|H_f^m(x) - H_f(x) \|_{\infty} \to 0 \quad (m \to \infty) 
\]

Build a sequence of polynomials $p_m(x)$ (of arbitrary degree) as follows. For $x_0 \in K$ (with respect to which $K$ is star-shaped), define polynomials $p_m(x)$ such that 

\begin{align*} 
&(1) \; p_m(x_0) = f(x_0) \\
&(2) \; \nabla p_m(x_0) = \nabla f(x_0) \\
& (3) \; H_{p_m}(x) = H_f^m(x) \quad \forall x \in K
\end{align*}

For completeness, one way to do this is as follows. Suppose we want a function $h$ such that $\nabla h = g$ and $h(x_0) = c$. Then we can set $h(x) = c + \int_0^1 g(\gamma_x(t)) \cdot \gamma_x'(t) \, dt$ for $x \in K$, where $\gamma_x$ is the line segment $[x_0,x]$. This type of construction allows us to build a polynomial of the form required above. \\

Given a polynomial $p_m$ that satisfies $(1) - (3)$ as above, it is simple to bound $\|f - p_m \|_{\infty}$. We include this calculation for the sake of completeness: \\

\tb{Convergence of functions from Hessian $\| \cdot \|_{\infty}$ convergence} - Choose $m$ such that we have $\|H_f^m - H_f \|_{\infty, K} < \eps$. Fix a path $\gamma(t)$ from $x_0 \to x$, which we will take to be the line segment $[x_0, x]$.  Define a function $\phi(t) = (f - p_m)(\gamma(t))$. Then by Taylor's Theorem, we have 

\begin{align*} 
&\phi(1) = \phi(0) + \phi'(0) + \frac{1}{2} \phi''(c) \quad c \in (0,1) \iff \\
& (f - p_m)(x) = (f-p_m)(x_0) + \nabla (f - p_m)(x_0)^T(x - x_0) + \frac{1}{2} (x - x_0)^T H_{f - p_m}(z) (x - x_0) 
\end{align*} 

Where $z \in [x_0, x]$. By construction of our polynomials, the first two terms are identically 0. Moreover, the component functions $\gamma(x)$ of the Hessian $H_{f - p_m}(x)$ have $\|\gamma\|_{\infty} < \eps$. Then by Gershgorin's Theorem, we have an eigenvalue bound

\[
\lambda(x) \leq \eps + \sum_{i \not = j} |\gamma_{ij}(x)| \leq p \cdot \eps
\]

Since the Hessian is in particular Hermitian at each point, this gives a 2-norm bound on the matrix $H_{f-p_m}$. Using Cauchy-Schwarz, we have that 

\[
2 |(f - p_m)(x)| \leq \|x - x_0 \|_2 \|H_{f - p_m}(x)(x - x_0)\|_2 \leq \|x - x_0\|_2^2 \cdot p \eps \leq Diam(K)^2 \cdot p \eps
\]

Then we have show that $\|H_f^m- H_f \|_{\infty} \to 0$ implies that $\|p_m - f(x) \|_{\infty} \to 0$. The $p_m$ are all sos-convex, so we have shown that $\overline{SOS} \supset SMSX$. \\

\tb{Approximation of Smooth Convex Functions} - We want to show that smooth strictly convex functions are dense in smooth convex functions i.e. $ \overline{SMSX} \supset SMX$. This is easy. Fix $\eps > 0$. Let $f \in SMX$. Then $f_{\eps}(x) := f(x) + \eps \|x \|_2^2$ is strictly convex and 

\[
|f_{\eps}(x) - f(x)| \leq \eps \|x\|_2^2 \leq \eps \cdot Diam(K)^2
\]

for$x \in K$. Then $\|f_{\eps} - f\|_{\infty} \to 0$ as $\eps \to 0$, so we have the density result. \\

\tb{Approximation of Continuous Convex Functions} - Note that convolution with a positive function preserves convexity. This is easily seen from the definition. Let $f$ convex and $\phi \geq 0$, then 

\[
(\phi \ast f)(x) = \int_{\mr^p} \phi(t) f(x - t) dt 
\]

Each $f_t(x) := f(x -t)$ is convex, therefore the convolution above is convex. \\

Let $\phi$ be (for instance) the pdf of a Gaussian $\mc{N}(0,I)$ (not actually necessary, we just need the function we are convolving against to be smooth and integrate to 1). For $\eps > 0$, define a class of functions $\phi_{\eps} = \frac{1}{\eps} \phi(\frac{x}{\eps})$. Fix $f$ a continuous convex function. If $f$ is not bounded, redefine $f = 0$ on $K^c$.  Define a class of functions $f_{\eps} := (\phi_{\eps} \ast f)$.\\

The standard arguments show that (i) $f_{\eps}$ is $C^{\infty}$ for each $\eps$ (ii) $f_{\eps} \to f$ uniformly on any compact set and (iii) $f_{\eps}$ is convex for all $\eps$. Then we have shown that $f$ is a limit point of $SMX$, therefore $\overline{SMX} \supset CTSX$. \\

\tb{Sketch of convolution argument} - I sketch the standard arguments for the sake of completeness. (iii) was shown above and basically follows from the definition of convexity. \\

For (i), note that 

\[
(\phi \ast f)(x) = \int_{\mr^p} \phi(t) f(x - t) dt = \int_{\mr^p} \phi(x - t) f(t) dt 
\]

so that whenever we need to take a derivative $\frac{\pa}{\pa x_{\alpha}} (\phi \ast f)(x)$, we just pass the differentiation operator through the integral (formal justification for this can be provided) and differentiate $\phi$, which is $C^{\infty}$. \\

For (ii), rewrite the integral above as 

\begin{align*}
f_{\eps} = (\phi_{\eps} \ast f)(x) &= \int_{\mr^p} \phi_{\eps}(t) f(x - t) dt\\
&=\int_{B_{\eps}} \phi_{\eps}(t) f(x - t) dt + \int_{B_{\eps}^c} \phi_{\eps}(t) f(x - t) dt
\end{align*}

Where we choose $B_{\eps}$ a very small ball containing $0$ of radius $h(\eps)$ such that (i) $\int_{B_{\eps}} \phi \approx 1$ and (ii) $\phi \approx 0 $ on $B_{\eps}^c$ as well as (iii) $f(x-t) \approx f(x)$ for $t \in B_{\eps}$ (this is where we use continuity on the class $CTSX$). Then roughly speaking the above decomposition is 

\begin{align*}
&\int_{B_{\eps}} \phi_{\eps}(t) f(x - t) dt + \int_{B_{\eps}^c} \phi_{\eps}(t) f(x - t) dt\\
&\approx f(x) \int_{B_{\eps}} \phi_{\eps}(t) dt + \int_{B_{\eps}^c} 0 \cdot f(x -t) dt \\
&\approx f(x) \cdot 1 + 0 = f(x)
\end{align*}

I claim that this gives uniform convergence $\|(\phi_{\eps} \ast f) - f\|_{\infty, K} \to 0$. Note that continuity of $f$ on $K$ implies uniform continuity, so the the approximation of the first integral in the above decomposition can be made uniform. $f$ is uniformly bounded on $\mr^p$, so the $2^{nd}$ integral above goes to $0$ uniformly. This sketch can be made rigorous but the arguments are relatively standard. \\

Then we have shown that \\

\[
SOSX \longrightarrow SMSX \longrightarrow SMX \longrightarrow CTSX
\]

where the arrow denotes density (in the sense of $\| \cdot \|_{K,\infty}$ for a fixed $K$ compact and star-shaped). In particular, we have shown that sos-convex polynomials are dense in continuous convex functions on a compact, convex set $K$. \\

\section{Convergence in Distribution}

Consider the shape-constrained maximum-likelihood estimator $f_n$ from the Cule paper. In this section, we show that there exists a a sequence of log sos-concave densities $p_n^m$ that converge in distribution to the Cule estimator $f_n$. \\

We also show that, as $d \to \infty$, the objective value, i.e. the achieved likelihood, of our program converges to the max-likelihood among log-concave estimators (the objective value of Cule's program). \\

Our approach will be as follows. Given the convex hull of the data $C_n$, we will construct a set $K = C_n^{\eps}$ such that $\mc{L}(C_n^{\eps} \setminus C_n) = \eps$, where $\mc{L}$ denotes Lebesgue measure. \\

We will then extend the concave function $s_n$ defining the Cule estimator ($f_n = \exp(s_n)$) to $K$ in such a way that (i) our extension is convex and (ii) our extension is increasingly negative on $\pa K$. Using our work above, let $p_m$ be a sequence of sos concave functions such that converge to our extension in $\| \cdot \|_{\infty}$. We will then argue that for any bounded function $G$ on $\mr^p$, we have 

\[
\int_{\mr^p} G(x) \exp(p_m(x)) = \int_{K^c} G(x) \exp(p_m(x)) + \int_{K \setminus C_n} G(x) \exp(p_m(x)) + \int_{C_n} G(x) \exp(p_m(x))
\]

We will show that (1) the first integral is small because loosely $p_m(x) \approx - \infty$ on $K^c$.  (2) The second integral is small because $\mc{L}(C_n^{\eps} \setminus C_n) \to 0$ as $\eps \to 0$. Finally, (3) the last integral converges to $\int_{C_n} G(x) f_n(x)$ because $\|p_m - s_n\|_{\infty,C_n} \to 0$. A rigorous formulation of this argument will show that for any bounded function $G$ we have 

\[
\int_{\mr^p} G(x) \exp(p_m(x)) dx \to \int_{\mr^p} G(x) f_n(x) dx \quad (m \to \infty) 
\]

i.e. $p_m$ converges to $f_n$ in distribution. \\

\tb{Definition of $K$} - Consider the tent function $s_n$ from Cule's paper. $s_n = -\infty$ outside of $C_n$. Note that as a convex hull, we have $C_n = \{Ax \leq b\}$ for some matrix $A \in \mr^{n \times p}$ and $b \in \mr^n$. Consider the set $C_n^{\eps} = \{Ax \leq b + \epsilon 1\}$. Note that $C_n^1$ is of finite Lebesgue measure since $C_n$ is [HOW?]. Then we can apply elementary measure theory to show that 

\[
\mc{L}(C_n) = \mc{L} \lft \bigcap_{\eps > 0} C_n^{\eps} \rt = \lim_{\eps \to 0} \mc{L}(C_n^{\eps})
\]

Therefore, for instance, we can make $\mc{L}(C_n^{\eps} \setminus C_n)$ arbitrarily small by letting $\eps \to 0$. Let $K_1 \supset C_n$ a polytope of the above form such that $\mc{L}(K_1 \setminus C_n) < \eps$. From Cule, there exists a triangulation of $C_n$ such that $s_n$ is affine on each simplex in the triangulation. \\

Extend this triangulation to a triangulation of the polytope $K_1$ and extend $s_n$ to $\ov{s_n}$ on $K_1$ by defining piecwise affine functions such that $\ov{s_n} = 0$ on $\pa K_1$. Since $Vol(K_1 \setminus C_n)$ is arbitrarily small and $\ov{s_n}$ is bounded on $K_1$, this extension does not affect the value of the integral 

\[
\int_{\mr^p} G(x) \exp(s_n) 
\]

asymptotically as $\eps = Vol(K_1 \setminus C_n) \to 0$. Therefore, wlog, we will assume from now on that $s_n$ vanishes on $\pa C_n$. \\

\tb{Extension of $s_n$ to $K$} - Define $\wh{s_n}$ such that $\wh{s_n}(x) = s_n(x)$ for $x \in C_n$ and $\wh{s_n}(x) = 0$ on $C_n^c$. Then $s_n$ is continuous but is in general no longer concave. Define a function on $K$ by 

\[
g(x) = \wh{s_n}(x) - M \cdot d(x,C_n) 
\]

\emph{Claim} - There exists an $N$ such that $M \geq N$ implies that $g$ is concave on $K$. Note that the set distance term is identically $0$ on $C_n$ and $\wh{s_n}$ is identically $0$ on $C_n^c$. Moreover, the set distance term is concave, since $x \to d(x,S)$ is a convex function whenever $S$ is a convex set. \\

By concavity of the original Cule function $s_n$ on $C_n$, for each point $x \in C_n$, the subgradient set $\pa s_n(x)$ is non-empty. In fact, from Cule's work we know that $s_n$ has the form 

\begin{align*}
s_n(x) &= \sum_k (a_k^T x + b_k) I(C_n^k) \quad (x \in C_n) \\
&= - \infty \quad else
\end{align*}

Where $C_n^k$ is a triangulation of the convex hull $C_n^k$. We can show that \\

\emph{Lemma} - For a function of the form above, $x \in C_n^k \implies a_k \in \pa s_n(x)$. The only non-trivial part of the argument is dealing with points $x \in \pa C_n^k$. Proof omitted. \\

\tb{Subgradient argument} - To show that our extension $g(x)$ is concave, it suffices to show that $\pa g(x) \not = \emptyset$ for each $x \in K$. Then by concavity of the disjointly supported pieces of $g(x)$, it suffices to show that the subgradient condition is satisfied for each $x \in C_n$, $y \in K \setminus C_n$ and conversely. \\

Consider $x \in C_n$ and $y \in K \setminus C_n$. Let $p_{y^*}$ denote the projection of $y$ onto $C_n$. In our construction of $g(x)$, choose 

\[
M \geq N = \max_{k} \|a_k\|_2 
\]

i.e. over all vectors defining the piecewise affine function $s_n$. Then we can calculate as follows 

\begin{align*}
g(y) - g(x) &= g(y) - g(p_{y^*}) + g(p_{y^*}) - g(x) = -M \|y - p_{y^*}\| + g(p_{y^*}) - g(x) \\
&\leq -M \|y - p_{y^*}\| + \pa g(x)^T(p_{y^*} - x) = -M \|y - p_{y^*}\| + \pa g(x)^T(p_{y^*} - y +y - x) \\
&\leq 0 + \pa g(x)^T(y - x) 
\end{align*}

Note that the $1^{st}$ inequality follows by concavity of $g$ on $C_n$, while the $2^{nd}$ inequality follows from Cauchy-Schwarz and our choice of $M$. A similar argument can be used to show that subgradients of $g$ exist for $x \in K \setminus C_n$. Then $g$ is concave on $K$. \\

By construction, for $M \geq N$ we have that $g$ is a concave function. Note that $x \in \pa K \implies d(x,K) \geq \eps$. Choose $K = C_n^{\eps}$ as previously defined, where $M_{\eps} \cdot \eps = N_{\eps} \to \infty$ as $\eps \to 0$.\\

\tb{Convergence argument} - We argue that $\int_{K^c} \exp(p_m) \to 0$ as $m \to \infty$. Let $x \in K^c$. Let $\pi_x$ denote the projection of $x$ onto $C_n$. Then there exists a point $x_0 \in \pa K \cap [\pi_x, x]$. Consider a function $\gamma(t) = p_m(\nu(t))$, where $\nu$ is a parametrization of the line $[\pi_x, x]$. Let $\nu(0) = \pi_x$ and $\nu(a) = x_0$. Then there exists a $c$ such that \\

\begin{align*}
&-M \eps = \gamma(a) - \gamma(0) = \gamma'(c) \|\pi_x - x_0\|\\
&\implies \gamma'(c) \leq \frac{-M \eps}{\|\pi_x - x_0\|} \leq \frac{-M \eps}{Diam(K)} 
\end{align*}

Since $\gamma$ is concave and $c \in [0,a]$, we then also have $\gamma'(a) \leq \frac{-M \eps}{Diam(K)}$. Then Taylor implies that 

\begin{align*}
p_m(x) = \gamma(1) &= \gamma(a) + \gamma'(a) \|x_0 - x\| + \frac{1}{2} \gamma''(d) \|x_0 - x\|^2\\
&\leq -M \eps + \frac{-M \eps}{Diam(K)} \|x - x_0\| \leq \frac{-M \eps}{Diam(K)} \|x - x^*\|
\end{align*} 

Where $x^*$ is the projection of $x$ onto $K$. Let $\phi_m = p_m + M \eps $. Then we have 

\begin{align*}
\int_{K^c} \exp(p_m) \leq \exp(-M \eps) \int_{K^c} \exp(\phi_m) \leq \exp(-M \eps) \int_{K^c} \exp \lft \frac{-M \eps}{Diam(K)}\|x - x_0\| \rt 
\end{align*}

We chose $m$ such that $M(\eps) \eps \to \infty$ as $\eps \to 0$. Then in particular for large enough $m$ the integral is 

\[
\leq \int_{K^c} \exp \lft -\|x - x^*\| \rt  = \int_{K^c} \exp \lft -d(x,K) \rt 
\]

This integral clearly converges since $K$ is compact. Since $-M \eps \to \infty$ as $\eps \to 0$, for any bounded function $\|G\|_{\infty, \mr^p} \leq B$ we have that 

\[
\int_{K^c} G \cdot \exp (p_m)  \to 0 \quad (m \to \infty) 
\]

\tb{Summary of Integral Resuts} - Note that by construction, we have $\mc{L}(C_n^{\eps} \setminus C_n) \to 0$ as $\eps \to 0$. Moreover, we used density of sos-convex polynomials to choose $p_m$ such that $|p_m(x)| \leq |p_m(x) - g_m(x)| + |g_m(x)| \leq \epsilon + 0 \leq 1$ on $C_n^{\eps} \setminus C_n$. Therefore we easily have that 

\[
\int_{C_n^{\eps} \setminus C_n} G \cdot \exp(p_m) \leq \int_{C_n^{\eps} \setminus C_n} Be = Be \cdot \mc{L}(C_n^{\eps} \setminus C_n) \to 0
\]

For the final integral in the decomposition, we have 

\[
\int_{C_n} G \cdot \exp(p_m) \to \int_{C_n} G \cdot s_n
\]

by bounded convergence on a finite measure space, with bound \\

\[
\|G \exp(p_m) \|_{\infty, C_n} \leq B \cdot (\|s_n\|_{\infty, C_n} + 1) 
\]

 Note that the latter bound is fixed for all $m$. Therefore, for any bounded function $G$ we have that 

\[
\int_{\mr^p} G\cdot \exp(p_m) \to \int_{\mr^p} G \cdot \exp(s_n) 
\]

In particular 

\[
\int_{\mr^p} \exp(p_m) \to \int_{\mr^p} \exp(s_n ) = 1
\]

Let $z_m = \frac{\exp(p_m)}{\int_{\mr^p} \exp(p_m)} $. Then the argument above shows that for any bounded function $G$

\[
\int_{\mr^p} G\cdot z_m \to \int_{\mr^p} G \cdot f_n
\]

So that we have convergence in distribution $z_m \to f_n$. \\

\tb{Convergence of objective function value} - Consider a function $z_m$ as constructed above such that $\|z_m - f_n\|_{\infty} \leq \eps$. Then for data $\{x_i\}_{i = 1}^n$, we must have in particular that $z_m(x_i) \geq f_n(x_i) - \eps$. Therefore, if we let $L$ denote the non-parametric likelihood, we have 

\[
L(z_m) \geq L(f_n) - n \eps_m \to L(f_n) \quad (m \to \infty) 
\]

Since $z_m \in SOSX$, the function returned by our convex problem must do better, so in particular our objective value converges to Cule's objective value as we let $d \to \infty$ (our construction uses $z_m$ of potentially arbitrarily high log-degrees). \\

\tb{Convergence of Our Estimator in Distribution} - It remains to show that our estimator converges in distribution to the Cule estimator. Let our estimator for degree $d$ be denoted by $\be_d$. Then we know that $L(\be_d) \to L(f_n)$ as $d \to \infty$. We somehow need to use convergence of likelihood values + the shape constraint that $\beta_m$ is in particular log-concave. \\

\emph{Question} - Does log-concavity + likelihood convergence imply convergence in distribution? 

















\clearpage


\tb{Things that we had trouble proving / omitted } \\

** Show that we have a density relation $SOSX \longrightarrow SMSX$. This proof currently has problems. \\

** Prove or disprove that our estimator $\be_d$ converges in distribution to $f_n$. \\

** Need to show that $\mc{L}(C_n) < \infty \implies \mc{L}(C_n^1) < \infty $. Note that this immediately implies that $\eps \to \mc{L}(C_n^{\eps} \setminus C_n)$ is a continuous function by standard measure theory arguments - we just need the measure space to be finite. \\

**Argumentation for existence of appropriate subgradients on $K \setminus C_n$ omitted. 






\end{document}